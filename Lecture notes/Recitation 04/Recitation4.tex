\documentclass[12pt]{article}
\usepackage{geometry}
\geometry{letterpaper, left=22.5mm, right=22.5mm, top=30mm, bottom=30mm}
\geometry{letterpaper}
\usepackage{amsmath}
\usepackage{amssymb}
\usepackage{enumitem}
\usepackage{fancyhdr}
\usepackage{framed}
\usepackage{tikz}
\usepackage{mathpazo}
%\usepackage{charter}
%\usepackage{newcent}
\usepackage{indentfirst}
\usepackage{booktabs}
\usepackage{graphicx}
\usepackage{float}
\usepackage{makecell}
\usepackage{xcolor}
\usepackage{mdframed}
\usetikzlibrary{trees}
\pagestyle{fancy}
\usepackage{amsthm}
\theoremstyle{definition}
\newtheorem{definition}{Definition}[section]
\theoremstyle{property}
\newtheorem{property}{Property}[section]
\theoremstyle{assumption}
\newtheorem{assumption}{Assumption}[section]
\theoremstyle{example}
\newtheorem{example}{Example}[section]
\theoremstyle{comment}
\newtheorem{comment}{Comment}[section]
\newtheorem{theorem}{Theorem}[section]
\newtheorem{corollary}{Corollary}[theorem]
\newtheorem{lemma}[theorem]{Lemma}
\usepackage{lastpage}
\usepackage{wrapfig}
\usepackage{hyperref}
\usepackage{subcaption}
\usepackage{setspace}
\hypersetup{
colorlinks=true,
filecolor=green, 
urlcolor=blue,
}
\newcommand{\ROM}[1]
    {\MakeUppercase{\romannumeral #1}}
\fancyhead[L]{Econometrics \ROM{2}: Recitation 12 }%change each reci
\fancyhead[R]{Spring 2020}
\fancyfoot[C]{\thepage \hspace{1pt} / \pageref*{LastPage}}

\fancypagestyle{firstpage}{%
\fancyhf{}%
\renewcommand{\headrulewidth}{0mm}%
  \fancyfoot[C]{\thepage \hspace{1pt} / \pageref*{LastPage}}
}

\lhead{Introduction to Econometrics}

\rhead{Recitation 4}


\title{Introduction to Econometrics: Recitation 4}

\begin{document}
\linespread{1.25}
\author{Seung-hun Lee}
\date{October 18th, 2021 }
\maketitle

\section{Nonlinear regressors}
Not everything in real life is correlated in a linear fashion. For instance, production function are not usually linear with respect to its inputs (Cobb-douglas production function), Wage rises, but the rate at which it rises falls over time (Mincer equation, from Mincer (1974)\footnote{\scriptsize{Mincer, Jacob (1974) "Schooling, Experience, and Earnings", New York, National Bureau of Economic Research}}), and the effect of classroom size can differ depending on how many students are in the class to begin with(Lazear(2001)\footnote{\scriptsize{Lazear, Edward (2001) "Educational Production", \textit{Quarterly Journal of Economics}, 116(3): 777-803}}). \par\medskip
For correlations like this, resorting to linear regressors would not allow the regression model to have the best fit. This is where the nonlinear regressors come in. Regressions with nonlinear regressors allow us to graph relationship between our $Y$ variable and our $X$ variable that is not necessarily linear. When incorporating such regressors, the interpretation of each coefficient becomes trickier. In the next few subsections, I will walk through implementing and interpreting coefficients for regressors that are not linear. \par\medskip
\subsection{Quadratic terms}
Recall from the regression with single independent variable $Y= \beta_0 + \beta_1X_1+u$, $\beta_1$ coefficient means the marginal effect of $X_1$ on $Y$. Mathematically, there are two ways to show this. (For now, all usual assumptions hold)
\begin{itemize}
\item \textbf{Derivatives}: Take derivatives on $Y$ with respect to $X_1$, this gets us
\[
\frac{\partial Y}{\partial X_1} = \frac{\partial}{\partial X_1}[ \beta_0 + \beta_1X_1+u ] = \beta_1
\]
Given that the idea of derivatives captures how much our $Y$ variable changes with unit change in $X_1$, $\beta_1$ represents how much $Y$ responds to a one unit change in $X_1$
\item\textbf{Taking Differences}: Suppose that you raise the amount of $X_1$ by $\Delta x$. Now the change in the  amount of $Y$, denoted as $\Delta y$, as a result of this change is
\[
\beta_0 + \beta_1(X_1+\Delta x)+u = Y+\Delta y
\] 
By subtracting $Y= \beta_0 + \beta_1X_1+u$, I get
\[
\Delta y = \beta_1 \Delta x \implies \frac{\Delta y}{\Delta x} = \beta_1
\]
\end{itemize} \par\medskip
In this example, the marginal effect of $X$ on $Y$ is constant - it does not depend on $X$! However, there are many relationships that cannot be explained this way. For instance, it is most likely the case that the wage rises with age. However, they are not likely to be in a linear relationship. As one gets older, the rate at which wage increases with age decreases. This is where \textbf{polynomial regressors} can improve the fitting of the regression model. Let wage be $W$ and age be $X$. I now regress the following model.
\[
W = \beta_0 + \beta_1 X+ \beta_2 X^2+u
\]
Then, the marginal impact of age on wage can be captured by
\[
\frac{\partial W}{\partial X} = \beta_1 + 2\beta_2 X
\]\par\medskip
Note now that unlike before, there is a term that depends on $X$. This implies that the marginal effect of age on wage now is different depending on what $X$ variables we use (or age). When $\beta_2$ is positive (negative), then marginal impact of age on wage is higher (lower) for older people. Simply put, age rises at faster (slower) rate for older people. \par\medskip

\subsection{$\ln$ (natural logarithm) terms}
We might be interested in how the changes in the independent variable $X$ in terms of \textit{percentages}, not levels, affect changes in the dependent variable $Y$. This is where \textbf{log regressors} can be used. Log regressors captures the idea of percentage changes. To see why, recall from calculus the first order approximation. For any differentiable funtion $f$, and a very small change in $x$, the following relationship holds. 
\[
f(x+\Delta x) \simeq f(x)+f'(x)[(x+\Delta x) -x] 
\]
Define $y=f(x)=\ln{x}$. Then $f(x+\Delta x) = y+\Delta y$ and $f'(x)=\frac{1}{x}$. We then get 
\[
\Delta y = \frac{\Delta x}{x}\implies \ln{(x+\Delta x)}-\ln{x} \simeq \frac{\Delta x}{x} \implies \ln\left(1+\frac{\Delta x}{x}\right)\simeq\frac{\Delta x}{x}
\]
Therefore, the changes in log values allows us to capture changes in percentages, at least for very small amount of change. \par\medskip
There are three different types of regression to go over. For now, I will use $\log$ to refer to the logarithm with base $e$, so equivalent with $\ln$, 
\begin{itemize}
\item\textbf{lin-$\log$}:  Consider the model $Y=\beta_0 + \beta_1 \log{X}+u$. I now back out $\beta_1$ to see what this coefficient implies. I take a before and after approach by changing $X$ by $\Delta x$. Then the total amount of $Y$ would be $Y+\Delta y$. Formally, 
\[
Y+\Delta y = \beta_0 + \beta_1 \log(X+\Delta x)+u
\]
Subtract $Y=\beta_0 + \beta_1 \log{X}+u$ from this equation to get
\[
\Delta y = \beta_1 \log(X+\Delta x)-\beta_1\log{X} = \beta_1 \log\left(1+\frac{\Delta x}{X} \right) = \beta_1 \frac{\Delta x}{X}
\]
Therefore, the following relationship holds
\[
\beta_1 = \frac{\Delta y}{(\Delta x/X)}
\]
Note that the percentage change in $X$ is $\frac{\Delta x}{X} \times 100$. In words, change in $X$ by 1 percent, raises $Y$ by $\beta_1 \times 0.01$. Put differently, 
\[
\Delta y = \beta_1 \times \frac{\Delta x}{X}
\]
Also, change in $X$ by 1\% implies that $\frac{\Delta x}{X}$ changes by 0.01. This is why we multiply 0.01 to $\beta_1$. 
\item\textbf{$\log$-lin}: Consider the model $\log{Y}=\beta_0 + \beta_1 X+u$. Conduct a similar before and after analysis as we did before to get the following result:
\[
\log(Y+\Delta y) = \beta_0 + \beta_1(X+\Delta x)+u
\]
Then, subtract $\log{Y}=\beta_0 + \beta_1 X+u$. This gets us
\[
\log\left(1+\frac{\Delta y}{Y}\right)\simeq \frac{\Delta y}{Y} = \beta_1 \Delta x
\]
Then, $\beta_1$ can be backed out as
\[
\beta_1 = \frac{(\Delta y / Y)}{\Delta x}
\]
Again, using the fact that percentage change in $Y$ can be represented as $\frac{\Delta y}{Y}\times 100$. This implies that a 1 unit change in $X$ raises $Y$ by $100\times \beta_1\%$. To see this, note that the above equation implies
\[
\frac{\Delta y}{Y} = \beta_1 \Delta x
\] 
By multiplying 100, the percentage change can be obtained. 
\item\textbf{$\log$-$\log$}: Consider the equation $\log{Y} = \beta_0 + \beta_1\log{X}+u$. Similar approach allows us to write
\[
\log(Y+\Delta y) = \beta_0 + \beta_1 \log(X+\Delta x)+u
\]
Subtract the original equation to obtain
\[
\log\left(1+\frac{\Delta y}{Y}\right)=\beta_1\log\left(1+\frac{\Delta x}{X}\right) \implies \frac{\Delta y}{Y}=\beta_1\frac{\Delta x}{X}
\]
which implies 
\[
\beta_1 = \frac{(\Delta y/Y)}{(\Delta x/X)}
\]
This implies that 1\% change in $X$ leads to $\beta_1\%$ change in $Y$. This is the \textit{elasticity} interpretation
\end{itemize}
\subsection{Interaction terms}
Suppose that we are interested in the relationship between test scores ($Y$) and class size($X_1$). However, one might guess that the effect of class size may differ depending on some other variables. For instance, one might guess that schools in districts where there are more funding ($X_2$) are more likely to enjoy the benefits of small school classroom\footnote{\scriptsize{This is inspired by the following work: \\ Lafortune, Julien, Jesse Rothstein, Diane W. Schanzenbach (2018) "School Finance Reform and the Distribution of Student Achievement", \textit{American Economic Journal: Applied Economics}, 10(2): 1-26 }}. In mathematical terms, the marginal effect of $X_1$ on $Y$ may depend on $X_2$. To capture this idea in a model, we can incorporate an \textbf{interaction term} involving $X_1$ and $X_2$, which can be written as $X_1 \times X_2$ \par\medskip
There are three types of interaction terms we can use: interaction between continuous variables, interaction between binary variables, and interaction with one binary and one continuous variable.
\begin{itemize}
\item\textbf{Binary $\times$ Binary: } Suppose that there are two binary variables, $D_1, D_2$. For example, let $D_1$ be union membership, and $D_2$ be workers working under fixed contract (hereafter regular-contract worker). The dependent variable of interest is wage. Notice the regression equation below
\[
Y= \beta_0 + \beta_1 D_1 + \beta_2 D_2 + \beta_3 (D_1 \times D_2) + u
\]  
By now, you know that $\beta_1$ captures the group difference in average for all workers with union membership and $\beta_2$ captures the same for all regular-contract workers. What do we want to do if we want to know the group difference in average for regular-contract workers with union membership? \par 
Note that $D_1 \times D_2$ becomes 1 only if the individual in question is a regular-contract worker with a union membership. This group average is captured by $\beta_3$ coefficient. To sum up:
\begin{itemize}
\item Union members: $E[Y|D_1 = 1] = \beta_0+\beta_1+\beta_2\times D_2+\beta_3\times D_2$
\item Non-members:  $E[Y|D_1 = 0] = \beta_0+\beta_2\times D_2$
\item Effect of union membership: $E[Y|D_1 = 1]-E[Y|D_1 = 0] = \beta_1 + \beta_3\times D_2$
\end{itemize}
So the effect of union membership differs depending on the worker being a regular-contract worker or not. We can see that this setup allows us to analyze the difference in effect of binary variable depending on another binary factor. 
\item \textbf{Binary $\times$ Continuous: } Instead of a second binary variable, we include a continuous variable $X_1$. Now we write
\[
Y= \beta_0 + \beta_1 D_1 + \beta_2 X_1 + \beta_3 (D_1 \times X_1) + u
\]
Let's go back to the example in the beginning of this subsection. In that example, classroom size would be a continuous variable, so that will be our $X_1$. Now, define $D_i=1$ if a school district receives funding and $0$ if otherwise. The effect of classroom size would now be 
\[
\frac{\partial Y}{\partial X_1} = \beta_2 + \beta_3 D_1
\]
Now the effect of $X_1$ depends on $D_1$. If the district receives the funding, the effect of classroom size would be $\beta_2 + \beta_3$. If the district does not get funding, the effect of classroom size would be $\beta_2$. By using this interaction term, we can analyze the effect of the continuous variable $X_1$ differently across different groups (who have different values of $D_i$). 
\item \textbf{Continuous $\times$ Continuous: } Consider this regression, where both $X_1$ and $X_2$ are continuous variables. 
\[
Y = \beta_0 + \beta_1X_1 + \beta_2 X_2 + \beta_3 (X_1\times X_2)+u
\] 
In this equation, the effect of $X_1$ on $Y$, and that for $X_2$ on $Y$ are
\[
\frac{\partial Y}{\partial X_1} = \beta_1 + \beta_3 X_2 \ \ \ \ \frac{\partial Y}{\partial X_2} = \beta_2 + \beta_3 X_1 
\]
Now you see that the marginal impact of $X_1$ on $Y$ is dependent of $X_2$. Similar can be said of marginal impact of $X_2$ on $Y$. An example would be letting $Y$ be wage, $X_1$ be age, and $X_2$ be years of experience. \par\medskip
\end{itemize}

%%%%%%%%%%%%%%%%
\end{document}


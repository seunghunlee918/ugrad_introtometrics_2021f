\documentclass[aspectratio=169]{beamer}

\mode<presentation>
\usetheme{Boadilla}
\definecolor{Columbia}{RGB}{185,217,235}
\definecolor{Columbia2}{RGB}{0,51,160}
\definecolor{Columbia3}{RGB}{0,114,206}
\definecolor{blue}{RGB}{30,90,205}
\definecolor{red}{RGB}{213,94,0}
\definecolor{green}{RGB}{0,128,0}
\setbeamercolor{title}{fg=Columbia3}
\setbeamercolor{frametitle}{fg=Columbia3}
\setbeamercolor{block title}{bg=Columbia3, fg=white}
\setbeamercolor{block body}{bg=white}
\setbeamercolor{structure}{fg=Columbia3}
\setbeamercolor{item projected}{fg=white}
\setbeamercolor{item}{fg=Columbia3}
\setbeamercolor{subitem}{fg=Columbia3}
\setbeamercolor{section in toc}{fg=Columbia3}
\setbeamercolor{description item}{fg=Columbia3}
\setbeamercolor{caption name}{fg=Columbia3}
\setbeamercolor{button}{bg=Columbia3, fg=white}
\usepackage{graphics}
\usepackage{geometry}
\usepackage{booktabs}
\usepackage{tikz}
\usepackage{amsmath}
\usepackage{bbm}
\usetikzlibrary{decorations.pathreplacing}
\usepackage{multirow, makecell}
\usepackage{float}
\usepackage{fancyvrb}
\usepackage{caption}
\usepackage{subcaption}
\usepackage{adjustbox}
\usepackage{threeparttable}
\usepackage{hyperref}
\usepackage[scaled=0.92]{helvet}
\newenvironment{wideitemize}{\itemize\addtolength{\itemsep}{10pt}}{\enditemize}
\newenvironment{wideenumerate}{\enumerate\addtolength{\itemsep}{10pt}}{\endenumerate}
\newenvironment{widedescription}{\description\addtolength{\itemsep}{10pt}}{\enddescription}
\hypersetup{
colorlinks=true,
linkcolor=blue,
filecolor=green, 
urlcolor=blue,
}
\beamertemplatenavigationsymbolsempty
\setbeamercolor{author in head/foot}{bg=white, fg=Columbia3}
\setbeamercolor{title in head/foot}{bg=white, fg=Columbia3}
\setbeamercolor{date in head/foot}{bg=white, fg=Columbia3}
\setbeamercolor{section in head/foot}{bg=white, fg=Columbia3}
\setbeamercolor{page number in head/foot}{bg=white, fg=Columbia3}
\setbeamercolor{headline}{bg=Columbia}
\setbeamertemplate{footline}{
    \leavevmode%
    \hbox{%
        \begin{beamercolorbox}[wd=.333333\paperwidth,ht=2.25ex,dp=1ex,center]{date in head/foot}%
            \usebeamerfont{date in head/foot}\insertshortdate
        \end{beamercolorbox}%
        \begin{beamercolorbox}[wd=.444444\paperwidth,ht=2.25ex,dp=1ex,center]{title in head/foot}%
            \usebeamerfont{title in head/foot}\insertshorttitle
        \end{beamercolorbox}%
        \begin{beamercolorbox}[wd=.222222\paperwidth,ht=2.25ex,dp=1ex, center]{page number in head/foot}%
            \usebeamerfont{page number in head/foot} \insertframenumber{} / \inserttotalframenumber
        \end{beamercolorbox}}%
        \vskip0pt%
    }
%\setbeamercolor{page number in head/foot}{fg=black}
\setbeamertemplate{section in toc}[sections numbered]
\setbeamertemplate{subsection in toc}{\leavevmode\leftskip=3em\rlap{\hskip-1.75em\inserttocsectionnumber.\inserttocsubsectionnumber}\inserttocsubsection\par}
\setbeamerfont{subsection in toc}{size=\footnotesize}
%\setbeamertemplate{headline}{%
  %\begin{beamercolorbox}[ht=5.5ex]{section in head/foot}
    %\vskip2pt\insertnavigation{0.33\paperwidth}\vskip2pt
  %\end{beamercolorbox}%
%}
\newenvironment{transitionframe}{\setbeamercolor{background canvas}{bg=Columbia3}\setbeamertemplate{footline}{} \begin{frame}}{\end{frame}}


\makeatletter
\let\@@magyar@captionfix\relax
\makeatother


\title[Recitation 10]{Recitation 10} % Change this regularly
\author[Seung-hun Lee]{Seung-hun Lee}
\institute[Columbia University]{Columbia University}

\date[December 13th, 2021]{December 13th, 2021}

\begin{document}
\begin{frame}
\titlepage
\end{frame}

\begin{frame}
\frametitle{Time series: Setting up the approach}
\begin{itemize}
\item Collect data on same observational unit $i$ for multiple time periods. 
\item Primary uses: Forecasting, modeling risks, and analyzing dynamic causal effects
\item Time series differs in that errors are likely to be autocorrelated and thus require different ways to calculate the standard error. 
\item Let $Y_t$ be the time series data captured at certain period $t$ - GDP
\item \textbf{Lags} are characterized as $Y_{t-1}$ and \textbf{leads} are defined as $Y_{t+1}$. 
\item $\Delta Y_{t}\equiv Y_t-Y_{t-1}$: The \textbf{first difference} at time $t$. 
\end{itemize}
\end{frame}

\begin{frame}
\frametitle{AR and ADL: Number of lags for dependent and independent variables }
\begin{itemize}
\item $AR(p)$: $Y_t$ is regressed against its own lagged values by $p$ times: 
\[
Y_t = \beta_0+\beta_1Y_{t-1}+...+\beta_pY_{t-p}+u_t 
\]
\begin{itemize}
\item Each coefficient $\beta_k$ indicates how past values are useful in forecasting
\end{itemize}
\item $ADL(p,q)$: $p$ lags of dependent variable and $q$ lags for $X$ variable
\[
Y_t = \beta_0+\beta_1Y_{t-1}+...+\beta_p Y_{t-p} + \delta_1 X_{t-1}+...+\delta_qX_{t-q}+u_t 
\]
\end{itemize}
\end{frame}

\begin{frame}
\frametitle{AR and ADL: Selecting number of lags}
\begin{itemize}
\item Right amount of $p$ and $q$ minimizes the following \textbf{information criteria}
\[
\begin{aligned}
AIC:& \ln\left(\frac{SSR(p,q)}{T}\right)+(K)\frac{2}{T}&
BIC:& \ln\left(\frac{SSR(p,q)}{T}\right)+(K)\frac{\ln{T}}{T}\\
\end{aligned}
\]
where $K=1+p+q$
\item \textbf{Granger causality}: Test that helps us see whether $X$ is useful in predicting $Y$
\[
H_0: \delta_1 = ... = \delta_q=0, \ \ H_1: \lnot H_0
\]
If the null hypothesis is rejected, we say that $X$ \textit{Granger-causes} $Y$
\end{itemize}
\end{frame}


\begin{frame}
\frametitle{Forecasting: What is a good forecast?}
\begin{itemize}
\item \textbf{Forecast inteval} captures how accurate your forecasts are. 
\item Assume an ADL(1,1) type of equation, with forecast error $Y_{T+1}-\hat{Y}_{T+1|T}$, or
\[
[(\beta_0-\hat{\beta}_0)+(\beta_1-\hat{\beta}_1)Y_{T}+(\delta_1-\hat{\delta}_1)X_T]+u_{T+1}
\] 
\item \textbf{Mean squared forecast error} (MSFE): standard errors used to create forecast interval
\[
\begin{aligned}
E[(Y_{T+1}-\hat{Y}_{T+1|T})^2]&=E[u_{T+1}^2]+E\left[\left((\beta_0-\hat{\beta}_0)+(\beta_1-\hat{\beta}_1)Y_{T}+(\delta_1-\hat{\delta}_1)X_T)\right)^2\right]\\
&=var(u_{T+1}) + var[(\beta_0-\hat{\beta}_0)+(\beta_1-\hat{\beta}_1)Y_{T}+(\delta_1-\hat{\delta}_1)X_T]
\end{aligned}
\]
\item As sample size increases, the uncertainty part (variance term), converges to 0. So MSFE is approximately equal to $var(u_{T+1})$. 
\end{itemize}
\end{frame}

\begin{frame}
\frametitle{Forecasting: How volatile is the forecast?}
\begin{itemize}
\item \textbf{Root mean squared forecast error} (RMSFE) is just $\sqrt{E[(Y_{T+1}-\hat{Y}_{T+1|T})^2]}$
\item Measures the spread of the forecast error distribution (magnitude of an `error in forecasting')
\item In practice, we estimate RMSFE using either SER (from several lectures ago) or based on actual forecast history for $t=t_1,..,T$ and get 
\[
MSFE = \frac{1}{T-t_1+1}\sum_{t=t_1-1}^{T-1}(Y_{t+1}-\hat{Y}_{t+1|t})^2
\]
\item \textbf{Forecast interval}: The bounds for the forecast intervals if size 95\% is constructed as
\[
\hat{Y}_{T+1|T}\pm 1.96\times \widehat{RMSFE}
\]
where $\widehat{RMSFE}$ is obtained using one of the two methods mentioned. 

\end{itemize}
\end{frame}

\begin{frame}
\frametitle{Stationarity: Is the distribution of the data stable across time or not?}
\begin{itemize}
\item Stationary: Distribution of $(Y_{t+1},..,Y_{t+s})$ does not depend on $t$. 
\item In other words, the distribution of $Y$ does not change over time
\item Nonstationary:  When there is a trend or a break in the movement of the data (or any change in underlying parameters), 
\item Trends
\begin{itemize}
\item \textbf{Deterministic trend} is a nonrandom  function of time, ($Y_t = \alpha t^2$)
\item \textbf{Stochastic trend} is random, and time-variant distribution, such as the random walk $Y_t = Y_{t-1}+u_t$  (You can check that $var(Y_t)=t\sigma_u^2$)
\item Any other case where $\beta_1>1$ is also nonstationary
\end{itemize} 
\end{itemize}
\end{frame}

\begin{frame}
\frametitle{Stationarity: Testing for this in AR(1)}
\begin{itemize}
\item Eyeball test: Check graphically
\item Dickey-Fuller test: Check for the existence of a `unit root' by testing
\[
H_0: \beta_1\geq 1,\ H_1 : \beta_1<1 
\]
\item For a general case with $AR(p)$, 
\[
\begin{aligned}
 Y_t&=\beta_0 + \beta_1 Y_{t-1}+ ...+\beta_pY_{t-p}+u_t\\
  &=\beta_0 + \beta_1Y_{t-1}\color{red}{-\beta_2 Y_{t-1}+\beta_2Y_{t-1}}\color{black}{+\beta_2 Y_{t-2}+...+\beta_pY_{t-p}+u_t}\\
  & ...\\
 &=\beta_0 + (\beta_1+...+\beta_p)Y_{t-1}-(\beta_2+...+\beta_p)\Delta Y_{t-1} - ... - \beta_{p}\Delta Y_{t-p+1}+u_t \\
 \end{aligned}
\]
\begin{itemize}
\item Test whether $H_0: \beta_1+...+\beta_p\geq 1,\ H_1 : \beta_1+...+\beta_p<1 $
\end{itemize}
\end{itemize}
\end{frame}

\begin{frame}
\frametitle{Stationarity: Testing structural breaks}
\begin{itemize}
\item Assume a ADL(1,1) structure, but that we know when the structural break occurs at year $\tau$ 
\item Let $D_t(\tau)=1$ if year $t\geq\tau$ and 0 otherwise. 
\item Then we write the equation as
\[
Y_t = \beta_0 +\beta_1Y_{t-1}+\delta_1 X_{t-1} + \gamma_1 D_t(\tau)+\gamma_2 D_t(\tau)Y_{t-1}+\gamma_3 D_t(\tau)X_{t-1}+u_t
\]
\item To check for structural break, test joint hypothesis of the following form:
\[
H_0: \gamma_1 = \gamma_2 = \gamma_3 =0, \ H_1: \lnot H_0
\]
This is the idea behind the \textbf{Chow test}. 
\item If structural break is unknown, we can do a \textbf{Quandt Likelihood Ratio test} that implements multiple Chow tests and finds the point where structural break most likely happened, if it occurred.  
\end{itemize}
\end{frame}

\begin{frame}
\frametitle{Dynamic Causal analysis with distributed lag models}
\begin{itemize}
\item \textbf{Dynamic causal effect} captures the effect of $X$ on $Y$ over time.
\item Write the distributed lag model as 
\[
Y_t = \alpha+\beta_0X_t + \beta_1X_{t-1}+...+\beta_pX_{t-p}+u_t
\]
\item $\beta_0$ captures the contemporaneous impact of $X$ on $Y$, holding past values of $X$ constant. 
\item $\beta_j , j\in[1,p]$ captures the impact of $X$ from $j$ period(s) ago on $Y$, holding $X$ from other periods constant
\end{itemize}
\end{frame}

\begin{frame}
\frametitle{Dynamic Causal analysis: Implementation}

\begin{itemize}
\item Cumulative effect: Cumulative effect can be captured by summing over multiple $\beta$'s
\item Specifically, we can write
\[
\begin{aligned}
Y_t& = \alpha+\beta_0X_t + \beta_1X_{t-1}+u_t\\
&=\alpha +\beta_0 X_t - \beta_0X_{t-1} + \beta_0 X_{t-1} + \beta_1 X_{t-1}+u_t \\
&=\alpha + \beta_0\Delta X_t + (\beta_0 + \beta_1)X_{t-1}+u_t\\
\end{aligned}
\]
\item Assumptions
\begin{itemize}
\item (Sequential) Exogeneity: $E[u_t|X_t, X_{t-1},...,X_1]=0$. Or that error terms should not be correlated with current and past values of $X$
\item Stationarity: $Y$ and $X$ should have stationary distributions and ($Y_t, X_t$) and ($Y_{t-j}, X_{t-j}$) becomes independent as $j$ gets large. 
\item $Y$ and $X$ has nonzero finite moments
\item There is no perfect multicollinearity
\end{itemize}
\end{itemize}
\end{frame}

\begin{frame}
\frametitle{Dynamic Causal analysis: Standard errors}

\begin{itemize}
\item Given that there is a possibility that autocorrelation can exist, we need a standard error that takes into account autocorrelation and heteroskedasticity. 
\item This is known as \textbf{heteroskedasticity and autocorrelation consistent} errors (HAC errors). 
\item The takeaway is that standard errors in the typical STATA output can be wrong and we need to take a slightly different approach.
\item Use \texttt{newey} in STATA, with $m$ lags for standard errors
\[
m = 0.75\times T^{1/3}
\]
where $T$ is the total time periods in the data
\end{itemize}
\end{frame}
%%%%%%%%%%%%% Section 1. 
%%%%%%%%%%%
\end{document}

\documentclass[aspectratio=169]{beamer}

\mode<presentation>
\usetheme{Boadilla}
\definecolor{Columbia}{RGB}{185,217,235}
\definecolor{Columbia2}{RGB}{0,51,160}
\definecolor{Columbia3}{RGB}{0,114,206}
\definecolor{blue}{RGB}{30,90,205}
\definecolor{red}{RGB}{213,94,0}
\definecolor{green}{RGB}{0,128,0}
\setbeamercolor{title}{fg=Columbia3}
\setbeamercolor{frametitle}{fg=Columbia3}
\setbeamercolor{block title}{bg=Columbia3, fg=white}
\setbeamercolor{block body}{bg=white}
\setbeamercolor{structure}{fg=Columbia3}
\setbeamercolor{item projected}{fg=white}
\setbeamercolor{item}{fg=Columbia3}
\setbeamercolor{subitem}{fg=Columbia3}
\setbeamercolor{section in toc}{fg=Columbia3}
\setbeamercolor{description item}{fg=Columbia3}
\setbeamercolor{caption name}{fg=Columbia3}
\setbeamercolor{button}{bg=Columbia3, fg=white}
\usepackage{graphics}
\usepackage{geometry}
\usepackage{booktabs}
\usepackage{tikz}
\usepackage{amsmath}
\usepackage{bbm}
\usetikzlibrary{decorations.pathreplacing}
\usepackage{multirow, makecell}
\usepackage{float}
\usepackage{fancyvrb}
\usepackage{caption}
\usepackage{subcaption}
\usepackage{adjustbox}
\usepackage{threeparttable}
\usepackage{hyperref}
\usepackage[scaled=0.92]{helvet}
\newenvironment{wideitemize}{\itemize\addtolength{\itemsep}{10pt}}{\enditemize}
\newenvironment{wideenumerate}{\enumerate\addtolength{\itemsep}{10pt}}{\endenumerate}
\newenvironment{widedescription}{\description\addtolength{\itemsep}{10pt}}{\enddescription}
\hypersetup{
colorlinks=true,
linkcolor=blue,
filecolor=green, 
urlcolor=blue,
}
\beamertemplatenavigationsymbolsempty
\setbeamercolor{author in head/foot}{bg=white, fg=Columbia3}
\setbeamercolor{title in head/foot}{bg=white, fg=Columbia3}
\setbeamercolor{date in head/foot}{bg=white, fg=Columbia3}
\setbeamercolor{section in head/foot}{bg=white, fg=Columbia3}
\setbeamercolor{page number in head/foot}{bg=white, fg=Columbia3}
\setbeamercolor{headline}{bg=Columbia}
\setbeamertemplate{footline}{
    \leavevmode%
    \hbox{%
        \begin{beamercolorbox}[wd=.333333\paperwidth,ht=2.25ex,dp=1ex,center]{date in head/foot}%
            \usebeamerfont{date in head/foot}\insertshortdate
        \end{beamercolorbox}%
        \begin{beamercolorbox}[wd=.444444\paperwidth,ht=2.25ex,dp=1ex,center]{title in head/foot}%
            \usebeamerfont{title in head/foot}\insertshorttitle
        \end{beamercolorbox}%
        \begin{beamercolorbox}[wd=.222222\paperwidth,ht=2.25ex,dp=1ex, center]{page number in head/foot}%
            \usebeamerfont{page number in head/foot} \insertframenumber{} / \inserttotalframenumber
        \end{beamercolorbox}}%
        \vskip0pt%
    }
%\setbeamercolor{page number in head/foot}{fg=black}
\setbeamertemplate{section in toc}[sections numbered]
\setbeamertemplate{subsection in toc}{\leavevmode\leftskip=3em\rlap{\hskip-1.75em\inserttocsectionnumber.\inserttocsubsectionnumber}\inserttocsubsection\par}
\setbeamerfont{subsection in toc}{size=\footnotesize}
%\setbeamertemplate{headline}{%
  %\begin{beamercolorbox}[ht=5.5ex]{section in head/foot}
    %\vskip2pt\insertnavigation{0.33\paperwidth}\vskip2pt
  %\end{beamercolorbox}%
%}
\newenvironment{transitionframe}{\setbeamercolor{background canvas}{bg=Columbia3}\setbeamertemplate{footline}{} \begin{frame}}{\end{frame}}


\makeatletter
\let\@@magyar@captionfix\relax
\makeatother


\title[Recitation 4]{Recitation 4} % Change this regularly
\author[Seung-hun Lee]{Seung-hun Lee}
\institute[Columbia University]{Columbia University}

\date[October 18th, 2021]{October 18th, 2021}

\begin{document}
\begin{frame}
\titlepage
\end{frame}

%%%%%%%%%%%%% Section 1. 

\begin{frame}
\frametitle{Not everything in world is linearly related}
\begin{itemize}
\item The effect of $X$ may grow larger/smaller as $X$ increases (think of wage and age)
 \item For correlations like this, nonlinear regressors are necessary
 \item When incorporating such regressors, the interpretation of each coefficient becomes trickier.
 \item Quadratic relations: Think about wage and age - wages increase with age, but (usually) at a decreasing pace
 \[
W = \beta_0 + \beta_1 X+ \beta_2 X^2+u
\]
\item The marginal effect of $X$ on $W$ can be written as
\[
\frac{\partial W}{\partial X}=\beta_1+2\beta_2 X
\]
\end{itemize}
\end{frame}

\begin{frame}
\frametitle{Back to linear models: So how is the marginal effect different?}
\begin{itemize}
 \item In a linear regressor only format, the marginal effect is
 \[
W = \beta_0 + \beta_1 X+u
\]
\[
\frac{\partial W}{\partial X}=\beta_1
\]
\item The difference is that with quadratic terms, we can express cases where \textit{marginal} changes to $W$ with respect to $X$ is not a constant, but depends on some value of $X$
\begin{itemize}
\item In the above case, if $\beta_2>0$, marginal increase in $W$ increases with $X$ (and vice versa)
\end{itemize}
\end{itemize}
\end{frame}


\begin{frame}
\frametitle{We are interested in percent changes (because measurement units)...}
\begin{itemize}
\item We might be interested in how the changes in the $X$ in terms of \textit{percentages} affect changes in the dependent variable $Y$. 
\item (Natural) Log regressors captures the idea of percentage changes. 
\item Recall from calculus the first order approximation: For any differentiable function $f$, and a very small change in $x$, the following relationship holds. 
\[
f(x+\Delta x) \simeq f(x)+f'(x)[(x+\Delta x) -x] 
\]
\item Define $y=f(x)=\ln{x}$. Then $f(x+\Delta x) = y+\Delta y$ and $f'(x)=\frac{1}{x}$. We then get 
\[
\Delta y = \frac{\Delta x}{x}\implies \ln{(x+\Delta x)}-\ln{x} \simeq \frac{\Delta x}{x} \implies \ln\left(1+\frac{\Delta x}{x}\right)\simeq\frac{\Delta x}{x}
\]
\item Therefore, the changes in log values allows us to capture changes in percentages, at least for very small amount of change.
\end{itemize}
\end{frame}

\begin{frame}
\frametitle{How much \textit{level} changes in $Y$ due to \textit{percentage} changes in $X$?}
\begin{itemize}
\item\textbf{lin-$\log$}:  Consider the model $Y=\beta_0 + \beta_1 \log{X}+u$.
\item I take a before and after approach by changing $X$ by $\Delta x$. 
\item Then the total amount of $Y$ would be $Y+\Delta y$. Formally, 
\[
Y+\Delta y = \beta_0 + \beta_1 \log(X+\Delta x)+u
\]
Subtract $Y=\beta_0 + \beta_1 \log{X}+u$ from this equation to get
\[
\Delta y = \beta_1 \log(X+\Delta x)-\beta_1\log{X} = \beta_1 \log\left(1+\frac{\Delta x}{X} \right) = \beta_1 \frac{\Delta x}{X}
\]
\item Therefore, 
\[
\beta_1 = \frac{\Delta y}{(\Delta x/X)}
\]
Note that the percentage change in $X$ is $\frac{\Delta x}{X} \times 100$. In words, change in $X$ by 1 percent, raises $Y$ by $\beta_1 \times 0.01$. 
\end{itemize}
\end{frame}

\begin{frame}
\frametitle{How much \textit{percentage} changes in $Y$ due to \textit{level} changes in $X$?}

\begin{itemize}
\item\textbf{$\log$-lin}: Consider the model $\log{Y}=\beta_0 + \beta_1 X+u$. 
\item Conduct a similar before and after analysis as we did before to get:
\[
\log(Y+\Delta y) = \beta_0 + \beta_1(X+\Delta x)+u
\]
\item Then, subtract $\log{Y}=\beta_0 + \beta_1 X+u$. This gets us
\[
\log\left(1+\frac{\Delta y}{Y}\right)\simeq \frac{\Delta y}{Y} = \beta_1 \Delta x
\]
\item Then, $\beta_1$ can be backed out as
\[
\beta_1 = \frac{(\Delta y / Y)}{\Delta x}
\]
Again, using the fact that percentage change in $Y$ can be represented as $\frac{\Delta y}{Y}\times 100$. This implies that a 1 unit change in $X$ raises $Y$ by $(100\times \beta_1)\%$. 
\end{itemize}
\end{frame}

\begin{frame}
\frametitle{How much \textit{percentage} changes in $Y$ due to \textit{percentage} changes in $X$?}
\begin{itemize}
\item\textbf{$\log$-$\log$}: Consider the equation $\log{Y} = \beta_0 + \beta_1\log{X}+u$. 
\item Similar approach allows us to write
\[
\log(Y+\Delta y) = \beta_0 + \beta_1 \log(X+\Delta x)+u
\]
\item Subtract the original equation to obtain
\[
\log\left(1+\frac{\Delta y}{Y}\right)=\beta_1\log\left(1+\frac{\Delta x}{X}\right) \implies \frac{\Delta y}{Y}=\beta_1\frac{\Delta x}{X}
\]
\item which implies 
\[
\beta_1 = \frac{(\Delta y/Y)}{(\Delta x/X)}
\]
This implies that 1\% change in $X$ leads to $\beta_1\%$ change in $Y$. This is the \textit{elasticity} interpretation
\end{itemize}
\end{frame}

\begin{frame}
\frametitle{IRL: Marginal effect of $X_1$ on $Y$ maybe a function of some other variables!}
\begin{itemize}
\item Suppose that we are interested in the relationship between test scores ($Y$) and class size($X_1$). 
\item However, one might guess that the effect of class size may differ depending on some other variables. 
\begin{itemize}
\item e.g. schools in districts where there are more funding ($X_2$) are more likely to enjoy the benefits of small school classroom
\end{itemize}
\item In math, the marginal effect of $X_1$ on $Y$ may depend on $X_2$. 
\item To capture this idea in a model, we can incorporate an \textbf{interaction term} involving $X_1$ and $X_2$, which can be written as $X_1 \times X_2$ \par\medskip
\end{itemize}
\end{frame}

\begin{frame}
\frametitle{Binary $\times$ Binary}
\begin{itemize}
\item Suppose that there are two binary variables, $D_1, D_2$.
\item Notice the regression equation below
\[
Y= \beta_0 + \beta_1 D_1 + \beta_2 D_2 + \beta_3 (D_1 \times D_2) + u
\]  
\item By now, you know that $\beta_1$ captures the group difference in average for those in group $D_1$ and $\beta_2$ captures the same for those in group $D_2$. 
\item Note that $D_1 \times D_2$ becomes 1 only individual is in both groups. The average for these individuals are captured by $\beta_3$ coefficient. 
\item To sum up:
\begin{itemize}
\item $E[Y|D_1 = 1] = \beta_0+\beta_1+\beta_2\times D_2+\beta_3\times D_2$
\item $E[Y|D_1 = 0] = \beta_0+\beta_2\times D_2$
\item $E[Y|D_1 = 1]-E[Y|D_1 = 0] = \beta_1 + \beta_3\times D_2$
\end{itemize}
\item So the effect of $D_1$ differs depending on $D_2$ as well. 
\item This setup allows us to analyze the difference in effect of binary variable depending on another binary factor. 
\end{itemize}
\end{frame}

\begin{frame}
\frametitle{Binary $\times$ Continuous}
\begin{itemize}
\item  Instead of a second binary variable, we include a continuous variable $X_1$. 
\item Now we write
\[
Y= \beta_0 + \beta_1 D_1 + \beta_2 X_1 + \beta_3 (D_1 \times X_1) + u
\]
\item In earlier example, classroom size would be a continuous variable, so that will be our $X_1$. 
\item Now, define $D_i=1$ if a school district receives funding and $0$ if otherwise. 
\item The effect of classroom size would now be 
\[
\frac{\partial Y}{\partial X_1} = \beta_2 + \beta_3 D_1
\]
\item Now the effect of $X_1$ depends on $D_1$ - whether the district receives funding or not
\end{itemize}
\end{frame}

\begin{frame}
\frametitle{Continuous $\times$ Continuous}
Interaction terms
\begin{itemize}
\item  Consider this regression, where both $X_1$ and $X_2$ are continuous variables. 
\[
Y = \beta_0 + \beta_1X_1 + \beta_2 X_2 + \beta_3 (X_1\times X_2)+u
\] 
\item In this equation, the effect of $X_1$ on $Y$, and that for $X_2$ on $Y$ are
\[
\frac{\partial Y}{\partial X_1} = \beta_1 + \beta_3 X_2 \ \ \ \ \frac{\partial Y}{\partial X_2} = \beta_2 + \beta_3 X_1 
\]
\item Now you see that the marginal impact of $X_1$ on $Y$ is dependent of $X_2$. 
\end{itemize}
\end{frame}


%%%%%%%%%%%
\end{document}

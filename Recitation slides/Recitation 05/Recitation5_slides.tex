\documentclass[aspectratio=169]{beamer}

\mode<presentation>
\usetheme{Boadilla}
\definecolor{Columbia}{RGB}{185,217,235}
\definecolor{Columbia2}{RGB}{0,51,160}
\definecolor{Columbia3}{RGB}{0,114,206}
\definecolor{blue}{RGB}{30,90,205}
\definecolor{red}{RGB}{213,94,0}
\definecolor{green}{RGB}{0,128,0}
\setbeamercolor{title}{fg=Columbia3}
\setbeamercolor{frametitle}{fg=Columbia3}
\setbeamercolor{block title}{bg=Columbia3, fg=white}
\setbeamercolor{block body}{bg=white}
\setbeamercolor{structure}{fg=Columbia3}
\setbeamercolor{item projected}{fg=white}
\setbeamercolor{item}{fg=Columbia3}
\setbeamercolor{subitem}{fg=Columbia3}
\setbeamercolor{section in toc}{fg=Columbia3}
\setbeamercolor{description item}{fg=Columbia3}
\setbeamercolor{caption name}{fg=Columbia3}
\setbeamercolor{button}{bg=Columbia3, fg=white}
\usepackage{graphics}
\usepackage{geometry}
\usepackage{booktabs}
\usepackage{tikz}
\usepackage{amsmath}
\usepackage{bbm}
\usetikzlibrary{decorations.pathreplacing}
\usepackage{multirow, makecell}
\usepackage{float}
\usepackage{fancyvrb}
\usepackage{caption}
\usepackage{subcaption}
\usepackage{adjustbox}
\usepackage{threeparttable}
\usepackage{hyperref}
\usepackage[scaled=0.92]{helvet}
\newenvironment{wideitemize}{\itemize\addtolength{\itemsep}{10pt}}{\enditemize}
\newenvironment{wideenumerate}{\enumerate\addtolength{\itemsep}{10pt}}{\endenumerate}
\newenvironment{widedescription}{\description\addtolength{\itemsep}{10pt}}{\enddescription}
\hypersetup{
colorlinks=true,
linkcolor=blue,
filecolor=green, 
urlcolor=blue,
}
\beamertemplatenavigationsymbolsempty
\setbeamercolor{author in head/foot}{bg=white, fg=Columbia3}
\setbeamercolor{title in head/foot}{bg=white, fg=Columbia3}
\setbeamercolor{date in head/foot}{bg=white, fg=Columbia3}
\setbeamercolor{section in head/foot}{bg=white, fg=Columbia3}
\setbeamercolor{page number in head/foot}{bg=white, fg=Columbia3}
\setbeamercolor{headline}{bg=Columbia}
\setbeamertemplate{footline}{
    \leavevmode%
    \hbox{%
        \begin{beamercolorbox}[wd=.333333\paperwidth,ht=2.25ex,dp=1ex,center]{date in head/foot}%
            \usebeamerfont{date in head/foot}\insertshortdate
        \end{beamercolorbox}%
        \begin{beamercolorbox}[wd=.444444\paperwidth,ht=2.25ex,dp=1ex,center]{title in head/foot}%
            \usebeamerfont{title in head/foot}\insertshorttitle
        \end{beamercolorbox}%
        \begin{beamercolorbox}[wd=.222222\paperwidth,ht=2.25ex,dp=1ex, center]{page number in head/foot}%
            \usebeamerfont{page number in head/foot} \insertframenumber{} / \inserttotalframenumber
        \end{beamercolorbox}}%
        \vskip0pt%
    }
%\setbeamercolor{page number in head/foot}{fg=black}
\setbeamertemplate{section in toc}[sections numbered]
\setbeamertemplate{subsection in toc}{\leavevmode\leftskip=3em\rlap{\hskip-1.75em\inserttocsectionnumber.\inserttocsubsectionnumber}\inserttocsubsection\par}
\setbeamerfont{subsection in toc}{size=\footnotesize}
%\setbeamertemplate{headline}{%
  %\begin{beamercolorbox}[ht=5.5ex]{section in head/foot}
    %\vskip2pt\insertnavigation{0.33\paperwidth}\vskip2pt
  %\end{beamercolorbox}%
%}
\newenvironment{transitionframe}{\setbeamercolor{background canvas}{bg=Columbia3}\setbeamertemplate{footline}{} \begin{frame}}{\end{frame}}


\makeatletter
\let\@@magyar@captionfix\relax
\makeatother


\title[Recitation 5]{Recitation 5} % Change this regularly
\author[Seung-hun Lee]{Seung-hun Lee}
\institute[Columbia University]{Columbia University}

\date[October 25th, 2021]{October 25th, 2021}

\begin{document}
\begin{frame}
\titlepage
\end{frame}

%%%%%%%%%%%%% Section 1. 

\begin{frame}
\frametitle{Critiquing the regressions}
\begin{itemize}
\item \textbf{External validity} is concerned with the applicability of the regression model to other contexts. 
\begin{itemize}
\item For instance, you may wonder whether regression model from schools in California could explain what happens in the classrooms in South Korea
\item Any critiques to the model questioning the replicability of the  result to other dataset is assessing the external validity. 
\end{itemize}
\item \textbf{Internal validity} assesses the model from the point of view of the population being studied. 
\begin{itemize}
\item This approaches questions the validity of the statistical inference within the given dataset. There are five threats to internal validity.
\end{itemize}
\end{itemize}
\end{frame}

\begin{frame}
\frametitle{Is the model itself set properly?}
\begin{itemize}
\item \textbf{Omitted variable bias: } Did the researcher left out factors that could affect $Y$ and are correlated with $X$? 
\item \textbf{Wrong functional form: } Did the researcher incorporate $X$ in a proper functional form? (quadratic, logs?)
\item \textbf{Errors in variables bias: } Measurement error. If $X$ does not have a clear cut measure, part of $X$ correlated to $Y$ and the observed version of $X$ is left out, resulting in attenuation bias
\item \textbf{Sample selection bias: }  If certain groups are more likely to not respond, then the data fails to represent population and the estimates based on this data is biased.
\item \textbf{Simultaneous causality bias: } There are cases where $Y$ causes $X$ too, also leading to the bias (consider supply and demand)
\end{itemize}
\end{frame}

\begin{frame}
\frametitle{Midterm Checklist}
\begin{itemize} 
\item To make your life easier, I will organize the list of things that you need to know for the upcoming midterm exam. 
\item Ideally, you would have looked back at the key proofs and derived them yourself, run loads of regressions to be familiar with OLS formula and a lot more.
\item If you are time-constrained, you are still okay. You can look at the items below and remind yourself of what you have learned.
\item Moreover, the list here is intended to give you some guidance of what you should review in the last minute (hopefullly). 
\item The most important things (things you should know by heart by now) are listed with $\star\star\star$. The more stars there are, the more crucial they are to understand what we have learned so far. 
\item However, it does not mean that the items with fewer stars can easily be ignored. I am giving some priority weights
\end{itemize}
\end{frame}

\begin{frame}
\frametitle{Concepts you should know}
\begin{itemize} 
\item[$\star\star\star$] \textbf{Hypothesis test}: Some people say that this is in the core of econometrics. By now you should know how to set up a null/alternative hypothesis, what type of test distribution to use ($F$, $t$, or normal), know your test statistic, and come to the conclusion using either the test statistics, p-value, or confidence interval. Also be familiar with $R^2$ and $\bar{R}^2$
\item[$\star\star\star$] \textbf{Interpreting coefficients}: The most important thing when you get your regression output is understanding what it means. Especially in multivariate case, you should know that the $\hat{\beta}$ coefficient in front of your independent variable is the marginal effect of changing $X$ by 1 unit on $Y$, \textit{holding other variables constant}. Interpretation is very crucial for nonlinear regressors. Be familiar with the regressions that involve logs and interaction terms. 
\item[$\star\star\star$]\textbf{Omitted variable bias}: This is what motivates the use of multivariate regressions. Do remember what the two conditions are and how you can mathematically show them.
\item[$\star\star\star$] \textbf{OLS assumptions}: Not that you need to memorize them. You do need to remind yourself how you proved unbiasedness, how you showed OVB, etc. with which assumptions. 
\end{itemize}
\end{frame}

\begin{frame}
\frametitle{Concepts you should know}
\begin{itemize} 
\item[$\star\star$] \textbf{Estimating the OLS}: Star docked off since you may be given the access to the formula during the exam. However, it is essential to know where the OLS estimators originated from. 
\item[$\star\star$] \textbf{Statistical properties of OLS}: Specifically, you need to be familiar with how to prove OLS estimator is unbiased. It does not mean you should memorize every line of the proof. It helps, however, to remind yourself which assumptions were used in the proof.
\item[$\star\star$]\textbf{Heteroskedasticity}: Remember how this affects $t$-statistics. Specifically, that it alters the standard errors but not the coefficients.  
\item[$\star\star$]\textbf{Multicollinearity}: Know two types of multicollinearity and what causes them. In particular, if you are given dummy variables, understand what to do with it. Specifically, \textit{If you have $K$ dummy variables for $K$ categories, include $K-1$ variables.}
\item[$\star\star$]\textbf{Binary (independent) variables}: Know how to set them up. Also, know that the coefficient in front of it implies the mean difference with the ``benchmark'' group.
\item[$\star\star$]\textbf{Internal/External Validity}: When the question asks you to assess the regression model and results, you can use these ideas to critique the model. 
\item[$\star$]\textbf{Reviewing Statistics}: This should have been taken care of as you moved onto the later chapters. If you are still not comfortable, look at this again as soon as possible. 
\end{itemize}
\end{frame}

\begin{frame}
\frametitle{Things to do}
\begin{itemize} 
\item[$\star\star\star$] \textbf{Do decent amount of studying, but definitely not too much}: Do get some enough sleep. It is never a good idea to cramp your econometrics knowledge overnight before the day of the exam. Always remember that slow and steady wins the race (especially for econometrics). 
\item[$\star\star\star$] \textbf{Use office hours}: I am talking about OH for both TA's and the professors. Our job is to help you get the most out of this course. We  are ready to help you in any way we can. Just visit remaining OH as often as possible and ask questions. It may just help to see what others are asking too.
\item[$\star\star\star$]\textbf{Review problem sets}: Not all of them. However, the ``pen(cil) and paper'' questions are a must. Since we assume that you are familiar with the problem set, it is important to walk back through the questions you solved before. 
\end{itemize}
\end{frame}

\begin{frame}
\frametitle{Things to do}
\begin{itemize} 
\item[$\star\star$] \textbf{Ungraded Problems in the problem set}: Some of them are empirical exercises. Nevertheless, they can also be a valuable source of new questions for those who need them. 
\item[$\star$]\textbf{End of the Chapter questions}: Since we did not cover some of the topics in the book, not all of the questions are relevant. Also note that the book approaches some matter differently from the class. Some of the questions that match what we learned, however, can be valuable for you. This is true if you need new questions to work on. 
\end{itemize}
\end{frame}
%%%%%%%%%%%
\end{document}
